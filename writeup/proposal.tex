\documentclass[11pt,]{article}
\usepackage[left=1in,top=1in,right=1in,bottom=1in]{geometry}
\newcommand*{\authorfont}{\fontfamily{pbk}\selectfont}
\usepackage[]{mathpazo}

\usepackage{framed}
\usepackage{parskip}

\usepackage[T1]{fontenc}
\usepackage[utf8]{inputenc}

\usepackage{abstract}
\renewcommand{\abstractname}{}    % clear the title
\renewcommand{\absnamepos}{empty} % originally center

\renewenvironment{abstract}
 {{%
    \setlength{\leftmargin}{0mm}
    \setlength{\rightmargin}{\leftmargin}%
  }%
  \relax}
 {\endlist}

\makeatletter
\def\@maketitle{%
  \newpage
%  \null
%  \vskip 2em%
%  \begin{center}%
  \let \footnote \thanks
    {\fontsize{18}{20}\selectfont\raggedright  \setlength{\parindent}{0pt} \@title \par}%
}
%\fi
\makeatother

\setlength{\parindent}{0pt}



\setcounter{secnumdepth}{0}

\usepackage{longtable,booktabs}


\title{Performance Persistence in Major League Baseball: Wharton Honors Thesis \thanks{Contact:
\href{mailto:chua@wharton.upenn.edu}{\nolinkurl{chua@wharton.upenn.edu}}}  }



\author{\Large Chris Hua\vspace{0.05in} \newline\normalsize\emph{Wharton School, University of Pennsylvania}   \and \Large Linda Zhao\vspace{0.05in} \newline\normalsize\emph{Wharton School, University of Pennsylvania}  }


\date{}

\usepackage{titlesec}

\titleformat*{\section}{\Large\bfseries}
\titleformat*{\subsection}{\Large\itshape}
\titleformat*{\subsubsection}{\normalsize\itshape}
\titleformat*{\paragraph}{\normalsize\itshape}
\titleformat*{\subparagraph}{\normalsize\itshape}




\newtheorem{hypothesis}{Hypothesis}
\usepackage{setspace}

\makeatletter
\@ifpackageloaded{hyperref}{}{%
\ifxetex
  \usepackage[setpagesize=false, % page size defined by xetex
              unicode=false, % unicode breaks when used with xetex
              xetex]{hyperref}
\else
  \usepackage[unicode=true]{hyperref}
\fi
}
\@ifpackageloaded{color}{
    \PassOptionsToPackage{usenames,dvipsnames}{color}
}{%
    \usepackage[usenames,dvipsnames]{color}
}
\makeatother
\hypersetup{breaklinks=true,
            bookmarks=true,
            pdfauthor={Chris Hua (Wharton School, University of Pennsylvania) and Linda Zhao (Wharton School, University of Pennsylvania)},
             pdfkeywords = {performance persistence, baseball, sports analytics},  
            pdftitle={Performance Persistence in Major League Baseball: Wharton Honors Thesis},
            colorlinks=true,
            citecolor=blue,
            urlcolor=blue,
            linkcolor=magenta,
            pdfborder={0 0 0}}
\urlstyle{same}  % don't use monospace font for urls

\usepackage{amsmath}

\providecommand{\tightlist}{%
  \setlength{\itemsep}{0pt}\setlength{\parskip}{0pt}}

\begin{document}
  
% \pagenumbering{arabic}% resets `page` counter to 1 
%
% \maketitle

{% \usefont{T1}{pnc}{m}{n}
\setlength{\parindent}{0pt}
\thispagestyle{plain}
{\fontsize{18}{20}\selectfont\raggedright 
\maketitle  % title \par  

}

{
   \vskip 13.5pt\relax \normalsize\fontsize{11}{12} 
 \textbf{\authorfont Chris Hua} \hfill \emph{\small Wharton School, University of Pennsylvania}   \par  \textbf{\authorfont Linda Zhao} \hfill \emph{\small Wharton School, University of Pennsylvania}   

}

}







\begin{abstract}

    \hbox{\vrule height .2pt width 39.14pc}

    \vskip 8.5pt % \small 

\noindent Lorem ipsum dolor sit amet, consectetur adipiscing elit. Morbi rhoncus
est metus, porttitor scelerisque nisi tincidunt at. Fusce pretium mi
nibh, pulvinar hendrerit turpis scelerisque nec. Etiam vitae auctor
erat, eget molestie massa. Morbi magna dolor, tincidunt quis iaculis et,
suscipit nec leo. Aenean et lectus lorem. Nullam suscipit eros et mi
eleifend, id eleifend enim ullamcorper. Aenean molestie vulputate urna,
non aliquet mi pellentesque eget.


\vskip 8.5pt \noindent \emph{Keywords}: performance persistence, baseball, sports analytics \par

    \hbox{\vrule height .2pt width 39.14pc}



\end{abstract}


\vskip 6.5pt

\noindent \onehalfspacing \section{Introduction}\label{introduction}

Performance persistence is a well-studied trend in the financial
literature, particularly involving mutual funds. In general, researchers
aim to determine if there is a cross-period effect where fund returns
can be predicted using past-period returns. The performance of sports
teams can be measured analogously to mutual fund returns, especially
when determining competitive equality.

Major League Baseball is an often studied sport in the academic
literature, in particular due to its wealth of data, mostly
individualistic nature, and large sample sizes.

\section{Literature Review}\label{literature-review}

We examine the literature here from two different perspectives: the
performance persistence literature from finance, and the competitive
equality literature in sports economics.

\subsection{Performance persistence}\label{performance-persistence}

Performance persistence has long been studied in the context of actively
managed funds, which attempt to select equities and other assets,
usually following some investment strategy, in order to maximize returns
to investors. In large part, the empirical findings have been mixed.

Performance is typically considered as a fund's ability to generate
`alpha', or more precisely, creating additional returns in excess of the
risk free rate, given the fund's market exposure. If we assume that the
Capital Asset Pricing Model (CAPM) holds, then alpha is simply given by
the slope of the following regression equation:

\[R_{p, t} - R_{f} = \alpha + \beta(R_{m,t} - R_f) + \epsilon_{p,t}\]
Rewritten in terms of alpha, we have:

\[\alpha = R_p - [R_f + \beta(R_{m,t} - Rf)]\]

However, this definition of alpha accounts only for risk attributable to
the market. Within a particular fund and their strategies, there are
additional sources of risk. Fama and French (1992) famously found that
much of the excess returns that practitioners believed was alpha was
actually attributable to exposure to various `factors,' namely market
capitalization (size) and `value' behavior (book-to-market ratio).

Carlson (1970) was one of the first papers in this area, and found
little predictive benefit from using the lagged returns.

Indeed, the importance of mutual funds and other actively managed funds
is widely accepted to be decreasing (see, e.g. Benjamin (2016)). The
mixed empirical evidence for performance-persistence is a strong driver
of this movement, as investors instead choose to invest in cheap
exposure to the market or particular strategies, rather than opt for an
expensive actively-managed portfolio.

\subsection{Competitive equality}\label{competitive-equality}

\section{Significance}\label{significance}

Good question

\section{Data}\label{data}

Calculations and writeups for this paper are done in the R language,
using the \texttt{RMarkdown} package for typesetting and reproducibility
in code (Xie 2014, Allaire et al. (2015)). Full code and writeup will be
available on the author's
\href{https://github.com/stillmatic/thesis}{Github}.

Major sports leagues have come to realize the importance of
comprehensive, open datasets. Major League Baseball in particular has
been on the forefront of the data revolution. At a high level, we do not
require particularly involved data, though. The most important data that
we require is number of games won at a per-team level, which is easily
found from a variety of sources, and should be easily available for all
major sports leagues.

In particular, we use the Sean Lahman database, and its \texttt{R}
interface, the package \texttt{Lahman} (Friendly 2016). This database
provides a comprehensive dataset of baseball statistics, and is easily
queryable.

Because the methodology we define is fairly generic and extensible, we
also identified other sources of data for use if we extend this analysis
to other sports. In particular,
\href{http://www.sports-reference.com/}{Sports Reference} provides a
number of useful data sources, including
\href{http://www.pro-football-reference.com/}{Pro Football Reference}
for the National Football League (NFL) and
\href{http://www.basketball-reference.com/}{Basketball Reference} for
the National Basketball Association (NBA).

\section{Methodology}\label{methodology}

There are several measures through which we measure repeat performance.
We outline:

\begin{enumerate}
\def\labelenumi{\arabic{enumi}.}
\tightlist
\item
  Contingency tables
\item
  Lagged regression
\end{enumerate}

\subsection{Contingency tables}\label{contingency-tables}

First, following (Brown and Goetzmann 1995) we use a nonparametric
contingency table-based methodology to measure repeat performance. We
define teams as ``winners'' or ``losers'' depending on a given metric.
Then, we measure the behavior of teams in a 2 year period, that is, they
are defined as ``winner-winner'' for 2014 if they are winners for 2014
and also winners in the 2015 season.

Then, we use the cross-product ratio to measure repeat performance.

\[R_{cp} = \frac{WW * LL}{WL*LW} \] We can approximate the standard
error of the natural log of the odds ratio {[}TODO: Christensen 1990
p40{]} as the following:

\[\sigma_{\ln R_{cp}} = \sqrt{WW^{-1} + WL^{-1} + LW^{-1} + LL^{-1}}\]

Then, we have sufficient framework to perform statistical tests of
significance.

\subsubsection{Hypotheses}\label{hypotheses}

First, we define teams as winners if they win more games than the median
number of games won per team for a given year.

\begin{framed}
\begin{hypothesis}
Performance in the first period is related to performance in the second period. 
\end{hypothesis}

\noindent $H_{0}^1$: Performance in the first period is unrelated to performance in the second period. That is, $R_{cp} = 1$. \\ 
$H_{1}^1$: Performance in the first period is related to performance in the second period. That is, $R_{cp} > 1$. 
\end{framed}

In the above sequence, we consider a team a winner by its performance
relative to the median winrate, which should be roughly 0.500, i.e.~50\%
winning rate. For the sake of comprehensiveness, we will also measure
team performance relative to the 0.500 benchmark.

\begin{framed}
\begin{hypothesis}
Performance in the first period is related to performance in the second period.
\end{hypothesis}

\noindent $H_{0}^2$: Performance in the first period is unrelated to performance in the second period. That is, $R_{cp} = 1$.  \\
$H_{1}^2$: Performance in the first period is related to performance in the second period. That is, $R_{cp} > 1$. 
\end{framed}

We also consider a performance measure where teams are considered
winners if they make the playoffs.

\begin{framed}
\begin{hypothesis}
Making the playoffs in the first period is related to making the playoffs in the second period.
\end{hypothesis}

\noindent $H_{0}^3$: Making the playoffs in the first period is unrelated to making the playoffs in the second period. That is, $R_{cp} = 1$.  \\
$H_{1}^3$: Making the playoffs in the first period is related to making the playoffs in the second period. That is, $R_{cp} > 1$. 
\end{framed}

Finally, to account for the peculiarities of American League vs National
League, the wild-card process, or general nonsense, we will also measure
winning rates relative to the ``worst'' team which does make the
playoffs, where worst is defined as fewest wins.

\begin{framed}
\begin{hypothesis}
Winning enough games to make the playoffs in the first period is related to winning enough games to make the playoffs in the second period.
\end{hypothesis}

\noindent $H_{0}^4$: Winning enough games to make the playoffs in the first period is unrelated to winning enough games to make the playoffs in the second period. That is, $R_{cp} = 1$.  \\
$H_{1}^4$: Winning enough games to make the playoffs in the first period is related to winning enough games to make the playoffs in the second period. That is, $R_{cp} > 1$. 
\end{framed}

\subsection{Lagged Regressions}\label{lagged-regressions}

A slightly more complex model for measuring effects over time is with a
lagged regression. That is, in a linear regression, is the winrate of
the year before a significant indicator for the winrate of the year
after? This is the approach taken by Carlson (1970).

For any given team \(i\) and season \(s\), we can then set up a
regression equation:

\[\pi_{i,s} = \beta_0 + \beta \pi_{i, s-1} + \epsilon\] Here,
\(\pi_{i,s}\) is the winrate from that season. Our goal is to measure
the significance of the \(\beta\) coefficient, which we can do through
an Anova test.

The most important benefit of using a regression model is that we can
control for exogenous effects, by adding them as additional factors into
the regression equation. Our results from a contigency table based
method may be better explained by other factors. Some factors which we
intend to control for include the following:

\begin{itemize}
\tightlist
\item
  \textbf{Player turnover}: We would expect a team with exactly the same
  players between two seasons to perform approximately the same.
  However, if teams have massive player turnover, then changes in
  performance may be due to the turnover rather than performance
  persistence. We can measure turnover by examining the number of
  players on the Opening Day roster that are not on the team's ending
  roster. We can also measure this change relative to other teams'
  turnover, and use a standardized z-score.
\item
  \textbf{Strength of schedule}: It is possible that some deviance in
  wins can be explained by how strong opponents were. This is typically
  measured by the aggregate number of games won by opponents. In the
  context of MLB teams, this is not likely to make a big difference,
  since over 162 games and many different opponents, it is unlikely to
  face large year-to-year changes in strength of schedule.
\item
  \textbf{Additional lagged years}: Our initial model is defined with 1
  year of lagged performance. We can also include more years for a more
  robust look at performance persistence.
\end{itemize}

Each of the above models described has been in the form of a linear
regression, that is, we have some continuous variable \(\pi\) which we
intend to solve for. We can also study this in the context of logistic
regression, which has a binary variable \(y\) to solve for; in this
case, we would use the previously described winner/loser classifications
for teams and then find the effect of past performance.

\subsubsection{Hypotheses}\label{hypotheses-1}

We begin with a simple base case.

\begin{framed}
\begin{hypothesis}
Given a regression of form $\pi_{i,s} = \beta_0 + \beta \pi_{i, s-1} + \epsilon$, one year's lagged performance is a significant regression variable.
\end{hypothesis}

\noindent $H_0^5$: Last season's performance is not a significant factor at the $p < 0.05$ level. \\
$H_1^5$: Last season's performance is a significant factor at the $p < 0.05$ level.
\end{framed}

We can also add more explanatory variables into the regression equation,
to control for endogenous effects. We outline a number of such variables
above. Let us call each such variable \(x_{i, j, s}\), where \(i\) is
the team, \(j\) is the effect number, and \(s\) is the season under
consideration.

\begin{framed}
\begin{hypothesis}
Given a regression of form $\pi_{i,s} = \beta_0 + \beta_{PP} \pi_{i, s-1} + \sum_{j}\beta_j x_{i, j, s} + \epsilon$, one year's lagged performance is a significant regression variable.
\end{hypothesis}

\noindent $H_0$: Last season's performance is not a significant factor at the $p < 0.05$ level. \\
$H_1$: Last season's performance is a significant factor at the $p < 0.05$ level.
\end{framed}

\subsection{Other adjustment factors}\label{other-adjustment-factors}

A fact of sport, and life, is that games are never fair. Bill James's
Pythagorean expectation for number of games won by a team, and its
variants, rely on the assumption that the number of games a team wins is
based on its quality, and the assumption that quality is proportional to
the ratio of runs scored over runs allowed. Better teams score far more
runs than they allow, so they win more games, and so on. In practice, we
use this measure to adjust for the effect of luck in win-loss records.
An unlucky team would lose in close games and win in blowouts.
Conversely, a lucky team would win more close games. Thus, there could
be value to using the Pythagorean expected wins rather than realized
wins as the dependent variable in a regression.

It is typically given by:
\[\hat\pi_{i, s} = \frac{\text{runs scored}^2}{\text{runs scored}^2 + \text{runs allowed} ^2}\]

Note that the above formula is easily generalizable to other competitive
sports.

To use the Pythagorean expectation, we will create a ``counterfactual
universe'', where each team's wins are given by their expected wins, and
our hypotheses are tested against this set of data.

\section{Data Analysis}\label{data-analysis}

Todo in full paper!

\section{Extensions}\label{extensions}

A core consideration in designing this analysis is how applicable these
methods are to other sports. While each sport obviously requires
different understandings, such as season length, each of these methods
can be applied without many changes. I think it will be within scope of
this paper to further incorporate data from other professional leagues
and compare the results.

\begin{longtable}[]{@{}lrrr@{}}
\toprule
League & Games Played & Number of Teams & Playoff Teams\tabularnewline
\midrule
\endhead
MLB & 162 & 30 & 10\tabularnewline
NFL & 16 & 32 & 12\tabularnewline
NBA & 82 & 30 & 16\tabularnewline
\bottomrule
\end{longtable}

It would be interesting to measure performance persistence at the
general manager or otherwise team management level. Brown and Goetzmann
(1995) finds support for performance persistence in mutual funds at the
fund manager level, which indicates the efficacy of common strategies
that managers carry along. With the increased coverage in baseball about
the importance of management who {[}Theo Epstein\ldots{}{]}

\section{Conclusion}\label{conclusion}

Todo in full paper!

\newpage

\section*{Bibliography}\label{bibliography}
\addcontentsline{toc}{section}{Bibliography}

\hypertarget{refs}{}
\hypertarget{ref-allaire2015}{}
Allaire, J, J Cheng, Yihui Xie, J McPherson, W Chang, Jeff Allen, H
Wickham, and R Hyndman. 2015. ``rmarkdown: Dynamic Documents for R.''
\emph{R Package Version 0.5}.

\hypertarget{ref-Benjamin2016}{}
Benjamin, Jeff. 2016. ``What's driving the decade of outflows from
actively managed mutual funds.''
\url{http://www.investmentnews.com/article/20160508/FREE/305089998/whats-driving-the-decade-of-outflows-from-actively-managed-mutual}.

\hypertarget{ref-Brown1995}{}
Brown, S, and William N. Goetzmann. 1995. ``Performance Persistence.''
\emph{The Journal of Finance} 50 (2): 679.
doi:\href{https://doi.org/10.2307/2329424}{10.2307/2329424}.

\hypertarget{ref-Carlson1970}{}
Carlson, Robert S. 1970. ``Aggregate Performance of Mutual Funds,
1948-1967.'' \emph{The Journal of Financial and Quantitative Analysis} 5
(1): 1--32. doi:\href{https://doi.org/10.2307/2979005}{10.2307/2979005}.

\hypertarget{ref-FamaFrench1992}{}
Fama, Eugene F., and Kenneth R. French. 1992. ``The cross-section of
expected stock returns.'' \emph{JoF} XLVII (2): 427--66.
doi:\href{https://doi.org/10.2307/2329112}{10.2307/2329112}.

\hypertarget{ref-Friendly2016}{}
Friendly, Michael. 2016. ``Sean 'Lahman' Baseball Database.''
Comprehensive R Archive Network (CRAN).
\url{https://cran.r-project.org/web/packages/Lahman/index.html}.

\hypertarget{ref-Xie2014}{}
Xie, Yihui. 2014. ``knitr: A Comprehensive Tool for Reproducible
Research in R.'' In \emph{Implementing Reproducible Research}, edited by
Victoria Stodden, Friedrich Leisch, and Roger D. Peng, 3--32. CRC Press.

\newpage
\singlespacing 
\end{document}
