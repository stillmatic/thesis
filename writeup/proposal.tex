ss[]{article}
\usepackage[left=1in,top=1in,right=1in,bottom=1in]{geometry}
\newcommand*{\authorfont}{\fontfamily{phv}\selectfont}
\usepackage[]{mathpazo}


  \usepackage[T1]{fontenc}
    \usepackage[utf8]{inputenc}



    \usepackage{abstract}
    \renewcommand{\abstractname}{}    % clear the title
    \renewcommand{\absnamepos}{empty} % originally center

    \renewenvironment{abstract}
     {{%
         \setlength{\leftmargin}{0mm}
             \setlength{\rightmargin}{\leftmargin}%
               }%
                 \relax}
                  {\endlist}

                  \makeatletter
                  \def\@maketitle{%
                    \newpage
                    %  \null
                    %  \vskip 2em%
                    %  \begin{center}%
                      \let \footnote \thanks
                          {\fontsize{18}{20}\selectfont\raggedright  \setlength{\parindent}{0pt} \@title \par}%
                          }
                          %\fi
                          \makeatother


                          

                          \setcounter{secnumdepth}{0}

                                                                                                                                  
                          
                                                    \title{Performance Persistence in Major League Baseball: Thesis Proposal - Wharton Honors Thesis  }
                          


                          \author{\Large Chris Hua\vspace{0.05in} \newline\normalsize\emph{Wharton School, University of Pennsylvania}  }


                          \date{}

                          \usepackage{titlesec}

                          \titleformat*{\section}{\normalsize\bfseries}
                          \titleformat*{\subsection}{\normalsize\itshape}
                          \titleformat*{\subsubsection}{\normalsize\itshape}
                          \titleformat*{\paragraph}{\normalsize\itshape}
                          \titleformat*{\subparagraph}{\normalsize\itshape}


                          
                                                    

                          \newtheorem{hypothesis}{Hypothesis}
                          \usepackage{setspace}

                          \makeatletter
                          \@ifpackageloaded{hyperref}{}{%
                          \ifxetex
                            \usepackage[setpagesize=false, % page size defined by xetex
                                          unicode=false, % unicode breaks when used with xetex
                                                        xetex]{hyperref}
                                                        \else
                                                          \usepackage[unicode=true]{hyperref}
                                                          \fi
                                                          }
                                                          \@ifpackageloaded{color}{
                                                              \PassOptionsToPackage{usenames,dvipsnames}{color}
                                                              }{%
                                                                  \usepackage[usenames,dvipsnames]{color}
                                                                  }
                                                                  \makeatother
                                                                  \hypersetup{breaklinks=true,
                                                                              bookmarks=true,
                                                                                          pdfauthor={Chris Hua (Wharton School, University of Pennsylvania)},
                                                                                                       pdfkeywords = {},  
                                                                                                                   pdftitle={Performance Persistence in Major League Baseball: Thesis Proposal - Wharton Honors Thesis},
                                                                                                                               colorlinks=true,
                                                                                                                                           citecolor=blue,
                                                                                                                                                       urlcolor=blue,
                                                                                                                                                                   linkcolor=magenta,
                                                                                                                                                                               pdfborder={0 0 0}}
                                                                                                                                                                               \urlstyle{same}  % don't use monospace font for urls

                                                                                                                                                                               

                                                                                                                                                                               \begin{document}
                                                                                                                                                                                   
                                                                                                                                                                                   % \pagenumbering{arabic}% resets `page` counter to 1 
                                                                                                                                                                                   %
                                                                                                                                                                                                                                                                                                                                                                      % \maketitle

                                                                                                                                                                                   {% \usefont{T1}{pnc}{m}{n}
                                                                                                                                                                                   \setlength{\parindent}{0pt}
                                                                                                                                                                                   \thispagestyle{plain}
                                                                                                                                                                                   {\fontsize{18}{20}\selectfont\raggedright 
                                                                                                                                                                                   \maketitle  % title \par  

                                                                                                                                                                                   }

                                                                                                                                                                                   {
                                                                                                                                                                                      \vskip 13.5pt\relax \normalsize\fontsize{11}{12} 
                                                                                                                                                                                      \textbf{\authorfont Chris Hua} \hskip 15pt \emph{\small Wharton School, University of Pennsylvania}   

                                                                                                                                                                                      }

                                                                                                                                                                                      }


                                                                                                                                                                                      
                                                                                                                                                                                      
                                                                                                                                                                                      
                                                                                                                                                                                                      \vskip 6.5pt

                                                                                                                                                                                                      \noindent  \subsection{Introduction}\label{introduction}

Performance persistence is a well-studied trend in the financial
literature, particularly involving mutual funds. In general, researchers
aim to determine if there is a cross-period effect where fund returns
can be predicted using past-period returns.

The performance of sports teams can be measured analogously to mutual
fund returns.

\subsection{Data and Methodology}\label{data-and-methodology}

Calculations and writeups for this paper are done in the R language,
using the RMarkdown package for typesetting and reproducibility in code
(Xie 2014, Allaire et al. (2015)).

\subsubsection{Data}\label{data}

Major sports leagues have come to realize the importance of
comprehensive, open datasets. Major League Baseball in particular has
been on the forefront of the data revolution. At a high level, we do not
require particularly involved data, though. The most important data that
we require is number of games won at a per-team level, which is easily
found from a variety of sources, and should be easily available for all
major sports leagues.

In particular, we use the

\subsubsection{Repeat performance
methodology}\label{repeat-performance-methodology}

There are several measures through which we measure repeat performance.

First, following (Brown and Goetzmann 1995) we use a nonparametric
contingency table-based methodology to measure repeat performance. We
define teams as ``winners'' or ``losers'' depending on if they win more
games than the median number of games won per team for a given year.
Then, we measure the behavior of teams in a 2 year period, that is, they
are defined as ``winner-winner'' for 2014 if they are winners for 2014
and also winners in the 2015 season.

Then, we use the cross-product ratio to measure repeat performance.

\[R_{cp} = \frac{WW * LL}{WL*LW} \]

\(H_{0}^1\): Performance in the first period is unrelated to performance
in the second period. That is, \(R_{cp} = 1\).\\
\(H_{1}^1\): Performance in the first period is related to performance
in the second period. That is, \(R_{cp} > 1\).

We can approximate the standard error of the natural log of the odds
ratio {[}TODO: Christensen 1990 p40{]} as the following:

\[\sigma_{\ln R_{cp}} = \sqrt{WW^{-1} + WL^{-1} + LW^{-1} + LL^{-1}}\]

In the above sequence, we consider a team a winner by its performance
relative to the median winrate, which should be roughly 0.500, i.e.~50\%
winning rate. For the sake of comprehensiveness, we will also measure
team performance relative to the 0.500 benchmark.

\(H_{0}^2\): Performance in the first period is unrelated to performance
in the second period. That is, \(R_{cp} = 1\).\\
\(H_{1}^2\): Performance in the first period is related to performance
in the second period. That is, \(R_{cp} > 1\).

We also consider a performance measure where teams are winners if they
make the playoffs.

\(H_{0}^3\): Making the playoffs in the first period is unrelated to
making the playoffs in the second period. That is, \(R_{cp} = 1\).\\
\(H_{1}^3\): Making the playoffs in the first period is related to
making the playoffs in the second period. That is, \(R_{cp} > 1\).

Finally, to account for the peculiarities of American League vs National
League, the wild-card process, or general nonsense, we will also measure
winning rates relative to the ``worst'' team which does make the
playoffs, where worst is defined as fewest wins.

\(H_{0}^4\): Winning enough games to make the playoffs in the first
period is unrelated to winning enough games to make the playoffs in the
second period. That is, \(R_{cp} = 1\).\\
\(H_{1}^4\): Winning enough games to make the playoffs in the first
period is related to winning enough games to make the playoffs in the
second period. That is, \(R_{cp} > 1\).

\subsection*{Bibliography}\label{bibliography}
\addcontentsline{toc}{subsection}{Bibliography}

\hypertarget{refs}{}
\hypertarget{ref-allaire2015}{}
Allaire, J, J Cheng, Yihui Xie, J McPherson, W Chang, Jeff Allen, H
Wickham, and R Hyndman. 2015. ``rmarkdown: Dynamic Documents for R.''
\emph{R Package Version 0.5}.

\hypertarget{ref-Brown1995}{}
Brown, S, and William N. Goetzmann. 1995. ``Performance Persistence.''
\emph{The Journal of Finance} 50 (2): 679.
doi:\href{https://doi.org/10.2307/2329424}{10.2307/2329424}.

\hypertarget{ref-Xie2014}{}
Xie, Yihui. 2014. ``knitr: A Comprehensive Tool for Reproducible
Research in R.'' In \emph{Implementing Reproducible Research}, edited by
Victoria Stodden, Friedrich Leisch, and Roger D. Peng, 3--32. CRC Press.

                                                                                                                                                                                                      \newpage
                                                                                                                                                                                                      \singlespacing 
                                                                                                                                                                                                                                                                                                                                                                                                                                                                                                                                                                                                                                                                                                                                                                                                                        \end{document}
